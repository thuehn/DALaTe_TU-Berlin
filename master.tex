\documentclass[
% papersize
a4paper,
%fond size
%12pt,
11pt,
%page print style
twoside,
%oneside,
openright]{book}

%for PSTRicks
\usepackage[pdf]{pstricks}
\usepackage[off]{auto-pst-pdf}

%graphics included tikz & pfg
\usepackage{graphicx}
\usepackage{epsfig,epic,eepic}
\usepackage{psfrag}
\usepackage[USenglish,german]{babel}
\usepackage{color,pstricks}
% To allow me to use letters with accent
\usepackage[utf8]{inputenc}

% Fonts
%\usepackage{lmodern}
%from Ruben \usepackage[T1]{fontenc}
\usepackage{times}
%\usepackage[scaled]{helvet}
%\usepackage[sc]{mathpazo} % math fonts for Palatino, see
                          % http://www.tug.dk/FontCatalogue/palatino/
%\usepackage{palatino}

% Math packages
\usepackage{amsmath,amssymb,amsfonts,amsthm}
\usepackage{url}
\usepackage[ruled,noline]{algorithm2e}
% More compact lists
\usepackage{paralist}
% Fullpage
\usepackage[in,headings]{fullpage}
% But then adapt spacing
\usepackage{setspace}
\setstretch{1.3}

%for nice tables
\usepackage{multirow}
%\usepackage{booktabs}
\usepackage{ctable}

%for nice item lists
\usepackage{paralist}

%usepackage comment for block comments
\usepackage{comment}

%nice symbols to draw arrow (yes) and -(no)
\usepackage{pifont}
\newcommand{\yes}{\checkmark}
\newcommand{\no}{\textendash}

%sub-picture numbering
%\usepackage{subcaption}
\usepackage{subfigure}

\usepackage[right]{eurosym}


%annotate TODO stuff
\usepackage[colorinlistoftodos, textwidth=4cm, shadow]{todonotes}
\usepackage{lscape}

%pseudocode package
\usepackage{pseudocode}

%to insert some clickable links
\usepackage{hyperref}
\definecolor{darkblue}{rgb}{0,0,.5}
\definecolor{black}{rgb}{0,0,0}
\hypersetup{colorlinks=true, breaklinks=true, citecolor=black, linkcolor=black, menucolor=black, urlcolor=black}

% \setlength{\oddsidemargin}{-0.3cm}
% \setlength{\evensidemargin}{-0.3cm}
% \setlength{\textwidth}{16cm}
% \setlength{\topmargin}{-0.8cm}
% \setlength{\textheight}{23.0cm}

% use some common definitions
\newcommand{\pref}[1]{Part~\ref{#1}}
\newcommand{\cref}[1]{Chapter~\ref{#1}}
\newcommand{\sref}[1]{Section~\ref{#1}}
\newcommand{\aref}[1]{Appendix~\ref{#1}}
\newcommand{\fref}[1]{Figure~\ref{#1}}
\newcommand{\tref}[1]{Table~\ref{#1}}
\newcommand{\eref}[1]{(\ref{#1})}

% More convenient sometimes
\newcommand{\ie}{i.\@e.\@ }
\newcommand{\eg}{e.\@g.\@ }

\newcommand{\us}{\,$\mu$s\xspace}

%add empty page macro
\newcommand{\blankpage}{
\newpage
\thispagestyle{empty}
\mbox{}
\newpage
}

% Some math definitions

%\newcommand{\indic}{1\hspace{-2.5mm}{1}}
%\newcommand{\indic}{\boldsymbol{1}}
\newcommand{\indic}[1]{1_{\{#1\}}}
\newcommand{\erfc}{\mathrm{erfc}}
\newcommand{\abs}[1]{\left\lvert #1 \right\rvert}
\newcommand{\lp}[1]{\left( #1 \right)}
\newcommand{\lb}[1]{\left\lbrack #1 \right\rbrack}
\newcommand{\lc}[1]{\left\lbrace #1 \right\rbrace}
\newcommand{\erf}[1]{\mathrm{erf}\lp{#1}}
\newtheorem{definition}{Definition}
\newtheorem{lemma}{Lemma}
\newtheorem{theorem}{Theorem}

\newcommand{\calI}{\mathcal I}
\newcommand{\calN}{\mathcal N}
\newcommand{\calP}{\mathcal P}
\newcommand{\calX}{\mathcal X}
\newcommand{\calC}{\mathcal C}

% Paper specific notation
\def\su{^{(u)}}
\def\s0{^{(0)}}
%
\def\Xr{X^{(r)}}
\def\Xt{X^{(t)}}
%
\def\Ns{N_{\textrm{s}}}
\def\Na{N_{\textrm{a}}}
%
\def\tacq{t_{\textrm{acq}}}
\def\ttx{t_{\textrm{tx}}}
\def\tfail{t_{\textrm{fail}}}
\def\tdrop{t_{\textrm{drop}}}
%
\def\tprop{t_{\textrm{prop}}}
\def\tpreamble{t_{\textrm{preamble}}}
\def\tdata{t_{\textrm{DATA}}}
\def\tack{t_{\textrm{ACK}}}
\def\tsigidle{t_{\textrm{SIGIDLE}}}
\def\tmaxbackoff{t_{\textrm{max backoff}}}
\def\tsendtimer{t_{\textrm{send timer}}}
\def\tidletimer{t_{\textrm{idle timer}}}
\def\tbackoff{t_{\textrm{backoff}}}
%
\def\SD{S_{\textrm{D}}}
\def\SI{S_{\textrm{I}}}
%
\def\Pacq{p_{\textrm{acq}}}
\def\Pfail{p_{\textrm{fail}}}
\def\Pbusy{P_{\textrm{busy}}}
\def\Pmd{P_{\textrm{MD}}}
\def\Pfa{P_{\textrm{FACQ}}}
%
\def\td{t_{\textrm{D}}}
\def\ti{t_{\textrm{I}}}
\def\Np{N_{\textrm{p}}}


\newcommand{\mand} {\mathrm{\;  and \; }}
\newcommand{\mfor} {\mathrm{\;  for \;  }}
\def\Ints{\mathbb{Z}}
\def\P{\mathbb{P}}
\def\E{\mathbb{E}}
\def\be{\begin{equation}}
\def\ee{\end{equation}}
\def\ben{\[}
\def\een{\]}
\def\bearn{\begin{eqnarray*}}
\def\eearn{\end{eqnarray*}}
\def\bear{\begin{eqnarray}}
\def\eear{\end{eqnarray}}
\def\barr{\begin{array}}
\def\earr{\end{array}}
\def\bel{\be \barr{l}}% equation array adjusted left, numbered
\def\eel{\earr\ee}
\def\beln{\ben \barr{l}}% equation array adjusted left, no number
\def\eeln{\earr\een}\def\be{\begin{equation}}
\def\ee{\end{equation}}
\def\ben{\[}
\def\een{\]}
\def\bearn{\begin{eqnarray*}}
\def\eearn{\end{eqnarray*}}
\def\bear{\begin{eqnarray}}
\def\eear{\end{eqnarray}}
\def\barr{\begin{array}}
\def\earr{\end{array}}
\def\bel{\be \barr{l}}% equation array adjusted left, numbered
\def\eel{\earr\ee}
\def\beln{\ben \barr{l}}% equation array adjusted left, no number
\def\eeln{\earr\een}


% Style headings + headers and footers

% Modify header and footers
\usepackage{fancyhdr}
\pagestyle{fancy}
\fancyhf{}
\renewcommand{\chaptermark}[1]{
  \markboth{\thechapter.\ #1}{}
%  \markboth{#1}{}
}
\renewcommand{\sectionmark}[1]{
%  \markboth{\thechapter.\ #1}{}
  \markright{#1}
}

% \fancyhead[LE]{\normalfont\sffamily\thepage}
% \fancyhead[RE]{\normalfont\sffamily\leftmark}
% \fancyhead[LO]{\normalfont\sffamily\rightmark}
% \fancyhead[RO]{\normalfont\sffamily\thepage}
\fancyhead[LE]{\normalfont\thepage}
\fancyhead[RE]{\normalfont\leftmark}
\fancyhead[LO]{\normalfont\rightmark}
\fancyhead[RO]{\normalfont\thepage}

% Modify the fonts for the chapters, section, sub... and paragraphs
\usepackage{titlesec}

\titleformat{\chapter}[display]
%{\normalfont\Huge\sffamily}
{\normalfont\Huge}
{\filleft{\textbf{\LARGE{\chaptername} {\thechapter}}}}{1em}{\filleft\textbf}

\titleformat{\section}
{\normalfont\Large}
{\filright{\textbf{\thesection}}}{1em}{\filright\textbf}

\titleformat{\subsection}
{\normalfont\normalsize}
{\filright{\textbf{\thesubsection}}}{1.3em}{\filright\textbf}

\titleformat{\subsubsection}
{\normalfont\normalsize}
{}{0em}{\filright\textbf}

\titleformat{\paragraph}[runin]
{\normalfont\normalsize}
{}{0em}{\filright\textbf}

% Do it for theorems,... too
%\theoremstyle{definition}
%\newtheorem{define}{\textbf\normalfont\normalsize\sffamily{Definition}}[chapter]
%\theoremstyle{plain}
%\newtheorem{theorem}{\textbf\normalfont\normalsize\sffamily{Theorem}}[chapter]
%\newtheorem{lem}{\textbf\normalfont\normalsize\sffamily{Lemma}}[chapter]


% The larger it is, the less willing LaTeX is to split footnotes
\interfootnotelinepenalty=10000


% Removing master, slows down the compilation
\includeonly{
	title,
	abstract,
	abstract_de,
	acknowledgments,
	paper-collaborators,
	eid,
	chapter-1_introduction,
	chapter-2_related_work,
	chapter-3_testbed,
	chapter-4_measurement_tools,
	chapter-5_algorithm,
	chapter-6_conclusion,
%	appendix,
}


% BEGINN of COCUMENT
\begin{document}

\setlength{\headheight}{13.6pt}

%TITLE PAGE
\pagestyle{empty}
% Titeseite

\begin{titlepage}
  \begin{center}
	  
    %\mbox{}

    %\vspace{1cm}

	\centering
	\includegraphics[width=3cm]{figures/tub.pdf}\\
	\vspace{0.8em}
	\LARGE 

	\textsc{Technische Universit\"at Berlin}

    \vspace{1cm}

    \sffamily \LARGE \textbf{My new Master thesis titel comes here}

    \vspace{1.5cm}





    \normalsize vorgelegt von

    \vspace{.1cm}

    \large Alex the student (Dipl.-Ing.)

    \vspace{.8cm}

  %  von der Fakult�t IV - Elektrotechnik und Informatik der Technischen Universit�t Berlin


    \normalsize von der Fakult\"{a}t IV -- Elektrotechnik und Informatik\\
    %\vspace{.1cm}
    \normalsize der Technischen Universit\"{a}t Berlin\\
    %\vspace{.1cm}
    \normalsize zur Erlangung des akademischen Grades\\
    %\vspace{.1cm}
    \large \textsc{Diplom der Ingenieurwissenschaften (Dipl.-Ing.)}\\
    %\vspace{.1cm}
    \normalsize genemigte Diplomarbeit\\

    \vspace{1cm}

    \large \textbf{Diplomausschuss:}

    \vspace{.2cm}

		\begin{tabular}{rl}
		Pr\"ufer der Diplomarbeit:	& Prof. Dr.-Ing. XYZ, TU-Berlin\\
						& Dr.-Ing. XYZ, TU-Berlin\\
		\end{tabular}\\

    \vspace{2cm}
    \large Tag der wissenschaftlichen Verteidigung: 11.01.2013\\
    \vspace{0.5cm}
    \large Berlin 2013\\

  \end{center}
\end{titlepage}


\selectlanguage{USenglish}

%ABSTRACT_EN
\frontmatter \pagestyle{plain}
\setcounter{page}{1}
%\addcontentsline{toc}{chapter}{Resume}
\chapter*{Abstract}
\label{chap:abstract}

%
To \textit{summarize} my thesis: 

%ABSTRACT_DE
\begin{otherlanguage*}{german}
  \addcontentsline{toc}{chapter}{Resume}
\chapter*{Zusammenfassung}
\label{chap:abstract_de}

%
Die \textit{Zusammenfassung} meiner Arbeit:

\end{otherlanguage*}

%ACKNOWLAGEMENTS
\addcontentsline{toc}{chapter}{Acknowledgments}
\chapter*{Acknowledgments}
\label{chap:ack}

%Acknowledgements for all the helping hands

\section*{}
At first, I want to thank ...





%PUBLICATIONS & COLLABORATORS
\chapter*{Pre-Published Papers}
\label{chap:prepublished_papers}

Parts of this thesis are based on the following peer-reviewed papers that have already been published. All my collaborators are among my co-authors.
%

%%%%%%%%%%%%%%%%%%%%%%%%%%%%%%%%%%%%%
%%%%%%%%%%%%%%%%%%%%%%%%%%%%%%%%%%%%%
%%%%%%%%%%%%   SECTION   %%%%%%%%%%%%
%%%%%%%%%%%%%%%%%%%%%%%%%%%%%%%%%%%%%
%%%%%%%%%%%%%%%%%%%%%%%%%%%%%%%%%%%%%
\section*{Journals}
\begin{list}{}
	{\leftmargin=2em \itemindent=-2em}
	\item 
		\textsc{Alex the student, Adam, Eve}: ``Challenges in Online Learning Platforms'', EURASTUD Journal on European Student Teaching,  2008, P. 1\hbox{-}10.
	%\item
\end{list}
\smallskip

%%%%%%%%%%%%%%%%%%%%%%%%%%%%%%%%%%%%%
%%%%%%%%%%%%%%%%%%%%%%%%%%%%%%%%%%%%%
%%%%%%%%%%%%   SECTION   %%%%%%%%%%%%
%%%%%%%%%%%%%%%%%%%%%%%%%%%%%%%%%%%%%
%%%%%%%%%%%%%%%%%%%%%%%%%%%%%%%%%%%%%
\section*{International Conferences}
\begin{list}{}
	{\leftmargin=2em \itemindent=-2em}
	\item 
		\textsc{Alex the student}: ``Practical Online Learning Tools'', in 21st International Student Teaching Conference (ISTC), 2012, P. 1\hbox{-}7.
	%\item
\end{list}
\smallskip

%%%%%%%%%%%%%%%%%%%%%%%%%%%%%%%%%%%%%
%%%%%%%%%%%%%%%%%%%%%%%%%%%%%%%%%%%%%
%%%%%%%%%%%%   SECTION   %%%%%%%%%%%%
%%%%%%%%%%%%%%%%%%%%%%%%%%%%%%%%%%%%%
%%%%%%%%%%%%%%%%%%%%%%%%%%%%%%%%%%%%%
\section*{Workshops}
\begin{list}{}
	{\leftmargin=2em \itemindent=-2em}
	\item 
	\textsc{Alex the student, Eve} : ``Online Learning Monitoring and Debugging through Measurement Visualization'', in the 4th IEEE International Workshop on Hot Topics in Online Learning (IEEE HotOnLe'12), 2012, P. 1\hbox{-}6.
	%\item
\end{list}
\smallskip


%%%%%%%%%%%%%%%%%%%%%%%%%%%%%%%%%%%%%
%%%%%%%%%%%%%%%%%%%%%%%%%%%%%%%%%%%%%
%%%%%%%%%%%%   SECTION   %%%%%%%%%%%%
%%%%%%%%%%%%%%%%%%%%%%%%%%%%%%%%%%%%%
%%%%%%%%%%%%%%%%%%%%%%%%%%%%%%%%%%%%%
\section*{Posters and Demos}
\begin{list}{}
	{\leftmargin=2em \itemindent=-2em}
	\item 
	\textsc{Alex the student, Adam}: ``OnTeMoS: Online Teacher Monitoring System'', in the 3rd International Workshop on Online Learning Approaches, Experimental Evaluation and Characterization, New York, NY, USA, 20013, P. 101\hbox{-}102.
	%\item
\end{list}

%EIDESSTATTLICHE ERKLAERUNG
%SELBSTAENDIGKEITSERKLAERUNG
\newcommand{\theauthor}{Alex the student}
\pagestyle{empty}

\vspace*{\stretch{1}}
\setlength{\parindent}{0in}
Ich versichere an Eides statt, dass ich diese Diplomarbeit selbst\"andig verfasst und nur die angegebenen Quellen und Hilfsmittel verwendet habe.
%
Alle w\"ortlich oder inhaltlich \"ubernommenen Stellen habe ich als solche gekennzeichnet.
%
Die vorliegende Arbeit wurde weder in der vorliegenden noch einer modifizierten Fassung einer dritten, in- oder ausl\"andischen Fakult\"at als Pr\"ufungsleistung, oder zum Erlangen eines akademischen Grades vorgelegt.

\vspace{1cm}

\begin{flushright}
	\begin{tabular}{p{2cm}p{6cm}}
		\hline
		Datum & \theauthor
		\smallskip
	\end{tabular}
\end{flushright}


%INSERT EMPTY PAGE
\blankpage

%TOC PAGE
\tableofcontents

%MAIN CONTENT
\mainmatter \pagestyle{fancy}
%\pagestyle{headings}
%%%%%%%%%%%%%%%%%%%%%%%%%%%%%%%%%%%%%%%%%%%%%%%%%%%%%%%%%%%%%%%%%%%%%%%%%%
%%%%%%%%%%%%   CAPTER 1   %%%%%%%%%%%%%%%%%%%%%%%%%%%%%%%%%%%%%%%%%%%%%%%%
%%%%%%%%%%%%%%%%%%%%%%%%%%%%%%%%%%%%%%%%%%%%%%%%%%%%%%%%%%%%%%%%%%%%%%%%%%
\chapter{Introduction}
\label{chap:introduction}

Introduction to the overall topic.

%%%%%%%%%%%%%%%%%%%%%%%%%%%%%%%%%%%%%
%%%%%%%%%%%%%%%%%%%%%%%%%%%%%%%%%%%%%
%%%%%%%%%%%%   SECTION   %%%%%%%%%%%%
%%%%%%%%%%%%%%%%%%%%%%%%%%%%%%%%%%%%%
%%%%%%%%%%%%%%%%%%%%%%%%%%%%%%%%%%%%%
\section{Problem Statement}
\label{sec:intro:probstatement}

The problem space is definced by the following.

%%%%%%%%%%%%%%%%%%%%%%%%%%%%%%%%%%%%%
%%%%%%%%%%%%%%%%%%%%%%%%%%%%%%%%%%%%%
%%%%%%%%%%%%   SECTION   %%%%%%%%%%%%
%%%%%%%%%%%%%%%%%%%%%%%%%%%%%%%%%%%%%
%%%%%%%%%%%%%%%%%%%%%%%%%%%%%%%%%%%%%
\section{Contributions}
\label{sec:intro:contrib}


Why is this an important problem ?
%

In summary, the main contributions of this thesis are:

\begin{itemize}
\item{\bf Tools}
  \begin{itemize}
    \item {\it Design and implementation of measurement tool \textit{Moni} to collect information:} In order to be able to analyze the impact, we needed direct measurements from the lower layer.
  \end{itemize}
\item {\bf Algorithm \& Implementation}
 	\begin{itemize}
    	\item {\it Linux sublayer extension to enable allocation per packet:} To enable setting per packet we extend Linux subsystem.
    	\item {\it Design and implementation of our joint controller}: We designed and implemented a joint controller within the Linux kernel.
	\end{itemize}
\item {\bf Measurements}
  \begin{itemize}
      \item {\it Feasibility of parameter control and its constraints:} We explore the capabilities of today's hardware and explored the ability to set different parameters.
	    \item {\it Validation and Performance Analysis:} We perform several validation experiments that confirm trobust operation.
  \end{itemize}
\end{itemize}

%%%%%%%%%%%%%%%%%%%%%%%%%%%%%%%%%%%%%
%%%%%%%%%%%%%%%%%%%%%%%%%%%%%%%%%%%%%
%%%%%%%%%%%%   SECTION   %%%%%%%%%%%%
%%%%%%%%%%%%%%%%%%%%%%%%%%%%%%%%%%%%%
%%%%%%%%%%%%%%%%%%%%%%%%%%%%%%%%%%%%%
\section{Thesis outline}
\label{sec:intro:outline}

The rest of this thesis is organized as follows:

\begin{compactitem}
  \item In Chapter 2, we discuss the theoretical and practical approaches.
  \item In Chapter 3, we present our Testbed in detail.
  \item In Chapter 4, we present our in-kernel monitoring tool .
  \item In Chapter 5, we present the design, the Linux implementation, the validation and performance evaluation of our controller.
  \item In Chapter 6, we summarize the contributions and limitations of our systems, and outline several directions for future work.
\end{compactitem}
%%%%%%%%%%%%%%%%%%%%%%%%%%%%%%%%%%%%%%%%%%%%%%%%%%%%%%%%%%%%%%%%%%%%%%%%%%
%%%%%%%%%%%%   CAPTER 2   %%%%%%%%%%%%%%%%%%%%%%%%%%%%%%%%%%%%%%%%%%%%%%%%
%%%%%%%%%%%%%%%%%%%%%%%%%%%%%%%%%%%%%%%%%%%%%%%%%%%%%%%%%%%%%%%%%%%%%%%%%%
\chapter{Related Work}
\label{chap:related_work}

%intro
%
To this end, we start with a short overview and then present the state-of-the-art of current approaches. Next, we discuss the current work.
%
Finally, we conclude with a summary and open problems.


%%%%%%%%%%%%%%%%%%%%%%%%%%%%%%%%%%%%%
%%%%%%%%%%%%%%%%%%%%%%%%%%%%%%%%%%%%%
%%%%%%%%%%%%   SECTION   %%%%%%%%%%%%
%%%%%%%%%%%%%%%%%%%%%%%%%%%%%%%%%%%%%
%%%%%%%%%%%%%%%%%%%%%%%%%%%%%%%%%%%%%
\section{A Short Overview}
\label{s:basics}

%\begin{figure}
%	\center
%	\includegraphics[width=\columnwidth]{figures/overview}
%	\caption{Overview.}
%	\label{fig:overview}
%\end{figure}


%%%%%%%%%%%%%%%%%%%%%%%%%%%%%%%%%%%%%
%%%%%%%%%%%%%%%%%%%%%%%%%%%%%%%%%%%%%
%%%%%%%%%%%%   SECTION   %%%%%%%%%%%%
%%%%%%%%%%%%%%%%%%%%%%%%%%%%%%%%%%%%%
%%%%%%%%%%%%%%%%%%%%%%%%%%%%%%%%%%%%%
\section{Control theory}
\label{s:single_control}

%
In this section, we discuss these approaches in their separate sections.
%

%%%%%%%%%%%%%%%%%%%%%%%%%
%%%%%   SUB-SECTION   %%%
%%%%%%%%%%%%%%%%%%%%%%%%%
%%%%%%%%%%%%%%%%%%%%%%%%%
\subsection{Old Control}
\label{ss:oc}


%summary table of old protocols
\begin{landscape}
\ctable[
	cap	= Summary of old Control (OC) Algorithms,
	caption = Summary of old Control (OC) Algorithms,
	label	= tab:oc_summary,
	pos  	= h
]{lccccc}{
%
\tnote[1]{Abbreviations: Obj: Objective, E: Energy, T: Topology, C: Capacity, Dist.: Distributed, Cent.:Centralized, synch: Synchronized, P: Prototype}
\tnote[2]{Old conditions are similar in space and time.}
\tnote[3]{Ideal comm: Overhearing any transmissions.}
\tnote[4]{New control considered.}
\tnote[5]{Angles between 5\textdegree to 120\textdegree are considered.}
%
} {
	\FL
	%
	\textbf{Protocol}			& \textbf{Obj.}\tmark[1]	& \textbf{Granularity}	& \textbf{Type of Control} & \textbf{PHY layer assumptions} & \textbf{Validation}\ML
	%
        PAR~\cite{Chen02-WNJ}			& E	& Per-packet	& Dist.	& Sym\tmark[2], ideal comm.\tmark[3],omni\tmark[4] & \no	\NN
        \hline	
	TER~\cite{154a}		& T	& Per-network	& Cent.	& Sym, ideal comm.,omni & \yes (Routing)	\NN
	MLP~\cite{aazhang88}		& T	& Per-cluster	& Cent.	& Sym, ideal comm.,omni & \yes (Routing)	\NN
        \hline	
	PCM~\cite{FuInfocom03}			& C	& Per-packet	& Dist.	& Sym, ideal comm., omni & \no	\NN
	POW~\cite{aiello03}	& C	& Per-packet	& Dist.	& Sym, ideal comm., omni & \no	\NN
	MID~\cite{callaway02}	& C	& Per packet	& Dist.	& Directional\tmark[5]	& \yes (MAC,WARP)	\NN
	DIR~\cite{bianchi00}			& C	& Per slot	& Cent.	& Directional	& \yes (MAC,Madwifi)	\NN
	SPE~\cite{crow97}			& C	& Per slot	& Cent.	& Directional	& \yes 	(Multi-radio)	\LL
}
\end{landscape}


In summary, it can be concluded that control does not bring energy savings.


%%%%%%%%%%%%%%%%%%%%%%%%%
%%%%%   SUB-SECTION   %%%
%%%%%%%%%%%%%%%%%%%%%%%%%
%%%%%%%%%%%%%%%%%%%%%%%%%
\subsection{New Control}
\label{ss:trc}

In this section, we summarize the algorithms in literature.
%
Table~\ref{tab:oc_summary} summarizes the approaches considered in this section.
%


%%%%%%%%%%%%%%%%%%%%%%%%%%%%%%%%%%%%%
%%%%%%%%%%%%%%%%%%%%%%%%%%%%%%%%%%%%%
%%%%%%%%%%%%   SECTION   %%%%%%%%%%%%
%%%%%%%%%%%%%%%%%%%%%%%%%%%%%%%%%%%%%
%%%%%%%%%%%%%%%%%%%%%%%%%%%%%%%%%%%%%
\section{Summary}

In this section, we gave an overview of the research on both theoretical and practical control.
%



%%%%%%%%%%%%%%%%%%%%%%%%%%%%%%%%%%%%%%%%%%%%%%%%%%%%%%%%%%%%%%%%%%%%%%%%%%
%%%%%%%%%%%%   CAPTER 3   %%%%%%%%%%%%%%%%%%%%%%%%%%%%%%%%%%%%%%%%%%%%%%%%
%%%%%%%%%%%%%%%%%%%%%%%%%%%%%%%%%%%%%%%%%%%%%%%%%%%%%%%%%%%%%%%%%%%%%%%%%%
\chapter{An Inside Look at Our Testbed}
\label{chap:testbed}



%%%%%%%%%%%%%%%%%%%%%%%%%%%%%%%%%%%%%
%%%%%%%%%%%%%%%%%%%%%%%%%%%%%%%%%%%%%
%%%%%%%%%%%%   SECTION   %%%%%%%%%%%%
%%%%%%%%%%%%%%%%%%%%%%%%%%%%%%%%%%%%%
%%%%%%%%%%%%%%%%%%%%%%%%%%%%%%%%%%%%%
\section{Our Testbed}
\label{sec:testbed}

In this section, we describe our testbed.

%%%%%%%%%%%%%%%%%%%%%%%%%
%%%%%   SUB-SECTION   %%%
%%%%%%%%%%%%%%%%%%%%%%%%%
%%%%%%%%%%%%%%%%%%%%%%%%%
\subsection{Hardware Setup}
\label{sec:testbed:hardware}



%%%%%%%%%%%%%%%%%%%%%%%%%
%%%%%   SUB-SECTION   %%%
%%%%%%%%%%%%%%%%%%%%%%%%%
%%%%%%%%%%%%%%%%%%%%%%%%%
\subsection{Software Setup}
%\paragraph{Software Components:}
\label{sec:testbed:software}
%


%%%%%%%%%%%%%%%%%%%%%%%%%%%%%%%%%%%%%
%%%%%%%%%%%%%%%%%%%%%%%%%%%%%%%%%%%%%
%%%%%%%%%%%%   SECTION   %%%%%%%%%%%%
%%%%%%%%%%%%%%%%%%%%%%%%%%%%%%%%%%%%%
%%%%%%%%%%%%%%%%%%%%%%%%%%%%%%%%%%%%%
\section{Characteristics}
\label{sec:characteristics}

In this section, we summarize characteristic operations.
%

%%%%%%%%%%%%%%%%%%%%%%%%%%%%%%%%%%%%%
%%%%%%%%%%%%%%%%%%%%%%%%%%%%%%%%%%%%%
%%%%%%%%%%%%   SECTION   %%%%%%%%%%%%
%%%%%%%%%%%%%%%%%%%%%%%%%%%%%%%%%%%%%
%%%%%%%%%%%%%%%%%%%%%%%%%%%%%%%%%%%%%
\section{Summary}
\label{sec:conclusion}

In this section, we gave an overview of our testbed.
%
\chapter{Measurement Tools}
\label{sec:mmeasurement}

To design, analyze, evaluate and debug  our controller,  a deep and precise understanding of the system behavior under different conditions is required.
%

%%%%%%%%%%%%%%%%%%%%%%%%%%%%%%%%%%%%%
%%%%%%%%%%%%%%%%%%%%%%%%%%%%%%%%%%%%%
%%%%%%%%%%%%   SECTION   %%%%%%%%%%%%
%%%%%%%%%%%%%%%%%%%%%%%%%%%%%%%%%%%%%
%%%%%%%%%%%%%%%%%%%%%%%%%%%%%%%%%%%%%
\section{Monitoring and its Limitations}
\label{sec:measurement:current}

%



%%%%%%%%%%%%%%%%%%%%%%%%%
%%%%%   SUB-SECTION   %%%
%%%%%%%%%%%%%%%%%%%%%%%%%
%%%%%%%%%%%%%%%%%%%%%%%%%
\subsection{Validation and Performance}
\label{sec:measurementg:validation}

In this section, we validate basic functions on different platforms to ensure its correct operation and analyze its limitations.
%


%%%%%%%%%%%%%%%%%%%%%%%%%%%%%%%%%%%%%
%%%%%%%%%%%%%%%%%%%%%%%%%%%%%%%%%%%%%
%%%%%%%%%%%%   SECTION   %%%%%%%%%%%%
%%%%%%%%%%%%%%%%%%%%%%%%%%%%%%%%%%%%%
%%%%%%%%%%%%%%%%%%%%%%%%%%%%%%%%%%%%%
\section{Summary}

In this chapter, we presented the design, validation and performance of our measurement tool.
%
\chapter{Development of an Algorithm}
\label{chap:controller}

In this chapter, we present the design of our controller, describe  our implementation, and discuss our validation and performance evaluation results.
%

%%%%%%%%%%%%%%%%%%%%%%%%%%%%%%%%%%%%%
%%%%%%%%%%%%%%%%%%%%%%%%%%%%%%%%%%%%%
%%%%%%%%%%%%   SECTION   %%%%%%%%%%%%
%%%%%%%%%%%%%%%%%%%%%%%%%%%%%%%%%%%%%
%%%%%%%%%%%%%%%%%%%%%%%%%%%%%%%%%%%%%
\section{Design and Implementation of a Control Interface}
\label{sec:controller:design}

In this section, we first give an overview of Linux framework, and next, describe the extensions we have implemented.
%


%%%%%%%%%%%%%%%%%%%%%%%%%
%%%%%   SUB-SECTION   %%%
%%%%%%%%%%%%%%%%%%%%%%%%%
%%%%%%%%%%%%%%%%%%%%%%%%%
\subsection{Linux framework}

In the next subsection, we describe the necessary restructuring of Linux framework to extend annotations to realize our controller.



%%%%%%%%%%%%%%%%%%%%%%%%%
%%%%%   SUB-SECTION   %%%
%%%%%%%%%%%%%%%%%%%%%%%%%
%%%%%%%%%%%%%%%%%%%%%%%%%
\subsection{Design Properties and Functionality}
\label{sec:controller:functions}

The algorithm operates in ine part: The pseudocode depicted in Fig.~\ref{fig:blues}) shows its operation.

%\newpage
%%%%%%%%%%%% PSYDOCODE of BLUES
\begin{figure}
 \begin{pseudocode}[ruled]{Blues}{ }
\GLOBAL{min\_update, min\_P, max\_P,\Delta_{inc},\Delta_{dec},\Delta,\delta_{inc},\delta_{dec}, \delta_{emergency}}\\
 \PROCEDURE{init}{ }
     \forall R:\\
        ref\_pwr[R], P\_data[R], P\_ack  \GETS max\_P; P\_sample[R] \GETS P\_ref[R]-\Delta\\
        \CALL{Init\_Stats}{}
\ENDPROCEDURE
\PROCEDURE{collect\_stats}{tx\textrm{-}feedback}
	%\COMMENT{For each rate R in the mrr retry chain}\\
	\FOREACH R \in tx\textrm{-}feedback.retry\_chain \DO\\
	\BEGIN
		%IF\ tx\textrm{-}feedback.got\_ACK(R) == true
		%\THEN
		%	success \GETS 1
		%\ELSE
		%	success \GETS 0\\
		success  \GETS tx\textrm{-}feedback.got\_ACK(R) == true\\
		\IF tx\textrm{-}feedback.power == ref\_power[R]
		\THEN
		%\BEGIN
			attempt\_ref[R]++; success\_ref[R]+=success;
		%\END
		\ELSEIF tx\textrm{-}feedback.power == data\_power[R]
		\THEN
		%\BEGIN
			attempt\_data[R]++; success\_data[R]+=success;
		%\END
		\ELSEIF tx\textrm{-}feedback\_power == sample\_power[R]
		\THEN
		%\BEGIN
			attempt\_sample[R]++; success\_sample[R]+=success;\\
		%\END\\
		\COMMENT{After sufficient samples, update Blues statistics}\\
		\IF attempts[sample\_pwr[R]] > update\_thresh \AND attempt\_ref[R] > update\_thresh	
		\THEN
			\CALL{Blues\_Update\_Stats}{R}\\
		\IF attempts[data\_pwr[R]] > update\_thresh
		\THEN
			\CALL{EWMA}{\alpha,p_{success}^{sample}[R], \frac{success\_sample[R]}{attempt\_sample[R]}}\\
	\END\\
	%\COMMENT{Sort all rates according to current throughput}\\
	sorted\_rate\_list \GETS \CALL{Sort\_by\_Throughput}{full\_rate\_set}\\
	sampling\_rate\_set \GETS subset[sorted\_rate\_set]		   
\ENDPROCEDURE

\PROCEDURE{Blues\_Update\_Stats}{R}
		\CALL{EWMA}{\alpha,p_{success}^{ref}[R], success\_ref[R/attempt\_ref[R]}\\
		\CALL{EWMA}{\alpha,p_{success}^{data}[R], success\_data[R]/attempt\_data[R]}\\
		\CALL{EWMA}{\alpha,p_{success}^{sample}[R], success\_sample[R]/attempt\_sample[R]}\\

		\COMMENT{If throughput collapse, reset to initial settings}\\
		\IF p_{success}^{data}[R] < \delta_{emergency} \OR
		current\_Thr[R] == 0
		\THEN
		%\BEGIN
			\CALL{Init}{}\\
			%\CALL{Reset\_Statistic}{R}\\
		%\END\\
		%
		\COMMENT{(1) Update  $sample\_power$,  $ref\_power$ and $data\_power$}\\
		\CALL{inc\_power}{P\_sample[R], p_{success}^{sample}[R], p_{success}^{ref}[R]}\\
		\CALL{dec\_power}{P\_sample[R], p_{success}^{data}[R], p_{success}^{ref}[R]}\\
		%
		%\COMMENT{(2) Update $ref\_power$}\\
		\CALL{inc\_power}{P\_ref[R], p_{success}^{ref}[R], 1}\\
		\CALL{dec\_power}{P\_ref[R], p_{success}^{ref}[R], 1}\\
		%
		%\COMMENT{(3) Update $data\_power$}\\
		P\_data[R] = P\_sample[R] +  \Delta
\ENDPROCEDURE
 \end{pseudocode}
\caption{Blues control algorithm}
\label{fig:blues}
\end{figure}



%%%%%%% Table of weighting  examples
\begin{landscape}
\ctable[
	cap	= Impact of different weight factors $w$ on the joint power and rate decision,
	caption = Impact of different weighting factors $w$ to the joint power and rate decision,
	label	= tab:weighting_fators,
	pos  	= t
]{lccccccccc}{
%
\tnote[1]{Throughput: Minstrel periodically estimates the achievable throughput per data rate based on measured success probabilities.}
\tnote[2]{Maximum utility at a given weighting factor $w$ is shown in bold.}
%
} {
	\FL
	%
	\multirow{2}{*}{\textbf{Data}}	& \multirow{2}{*}{\textbf{Thr.\tmark[1]}}	& \multirow{2}{*}{\textbf{Power}}	& \multirow{2}{*}{\textbf{Power}} & \multirow{2}{*}{\textbf{Benefit}} & \multirow{2}{*}{\textbf{Cost}}	& \multicolumn{4}{c}{\textbf{Utility}}	\NN
	%
	\textbf{Rate} &[MBit\/s] &[dBm] &[mW] &[\%] &[\%] & \textit{\small{$w=1$}}	&	\textit{\small{$w=2$}}	&	\textit{\small{$w=4$}}	&	\textit{\small{$w=10$}}	\NN
	%
	\cmidrule(rl){1-1}\cmidrule(rl){2-2}\cmidrule(rl){3-3}\cmidrule(l){4-4}\cmidrule(rl){5-5}\cmidrule(l){6-6}\cmidrule(l){7-10}
	%
	6	&	6.0		&	1	&	1.3		&	39.2	&	20.3	&	19.0			&29.1			&	34.1			&	37.2	\NN
	9	&	8.8		&	1	&	1.3		&	57.5	&	13.8	&	43.7			&50.6			&	54.1			&	55.1	\NN
	12	&	11.7	&	1	&	1.3		&	76.5	&	10.4	&	\textbf{66.1}\tmark[2]	&\textbf{71.3}\tmark[2]	&	73.9			&	75.4	\NN
	18	&	15.3	&	12	&	15.9	&	100.0	&	100.0	&	0.0				&50.0			&	\textbf{75.0}\tmark[2]	&	\textbf{90.0}\tmark[2]	\NN
	24	&	4.7		&	19	&	79.4	&	30.7	&	1631.5	&	-1600.0			&-758.0			&	-377.2			&	-132.4	\NN
	36	&	2.1		&	19	&	79.4	&	13.7	&	3651.5	&	-3600.0			&-1812.0		&	-899.2			&	-351.4	\NN
	48	&	0.0		&	19	&	79.4	&	0.0		&	0.0		&	0.0				&0.0			&	0.0				&	0.0		\NN
	54	&	0.0		&	19	&	79.4	&	0.0		&	0.0		&	0.0				&0.0			&	0.0				&	0.0		\LL
}

Table~\ref{tab:weighting_fators} provides a detailed example of benefit, cost, and utility calculations.
\end{landscape}

%%%%%%%%%%%%%%%%%%%%%%%%%%%%%%%%%%%%%
%%%%%%%%%%%%%%%%%%%%%%%%%%%%%%%%%%%%%
%%%%%%%%%%%%   SECTION   %%%%%%%%%%%%
%%%%%%%%%%%%%%%%%%%%%%%%%%%%%%%%%%%%%
%%%%%%%%%%%%%%%%%%%%%%%%%%%%%%%%%%%%%
\section{Performance Evaluation}
\label{sec:controller:evaluation}


In this section, we describe the performance in different scenarios, ranging from single link to two or more links that operate in parallel.
%


%%%%%%%%%%%%%%%%%%%%%%%%%%%%%%%%%%%%%
%%%%%%%%%%%%%%%%%%%%%%%%%%%%%%%%%%%%%
%%%%%%%%%%%%   SECTION   %%%%%%%%%%%%
%%%%%%%%%%%%%%%%%%%%%%%%%%%%%%%%%%%%%
%%%%%%%%%%%%%%%%%%%%%%%%%%%%%%%%%%%%%
\section{Summary}
\label{sec:controller:conclusion}

In this chapter, we presented the design, implementation, and performance evaluation of our controller. 
%
In the first part, we explained our modifications to the Linux subsystem .
%
Building upon this, we described our heuristic approach to minimize control decisions.
%

%
The design and implementation of our controller within the Linux kernel shows good performance.
%

In our performance evaluation in the testbed we showed that our controller performs well.
%
Our measurements cover three different szenarios.

%%%
\paragraph{Limitations:}

In this chapter, we analyzed our controller with its ability gain performance.
%
Our current experimentation setup does not cover mixed traffic, different packet sizes, or multiple packet flows and distributions.
%
Therefore, additional measurements would be necessary to explore the effects of realistic user traffic.
%

\chapter{Conclusion and Outlook}
\label{chap:conclusion}


%%%%%%%%%%%%%%%%%%%%%%%%%%%%%%%%%%%%%
%%%%%%%%%%%%%%%%%%%%%%%%%%%%%%%%%%%%%
%%%%%%%%%%%%   SECTION   %%%%%%%%%%%%
%%%%%%%%%%%%%%%%%%%%%%%%%%%%%%%%%%%%%
%%%%%%%%%%%%%%%%%%%%%%%%%%%%%%%%%%%%%
\section{Summary}

We started by asking three questions: (1) \textit{What} are the necessary prerequisites in order to achieve efficient rcontrol? (2) \textit{How} do we design and implement this control? (3) \textit{What} are the conditions that allow control to increase total performance?.

Chapter 3 and 4 answer the first question.
%

Chapter 5 aims at answering the remaining two questions on the design and implement of a controller.
%
In the first part, we explained our modifications to the Linux subsystem.
%
To the best of our knowledge, we are the first to implement such a controller. 
%
Building upon this, we described our heuristic approach to minimize control decisions.
%


Our implementation takes into account different capabilities, e.g., data controllability.

Finally, based on our interface, we designed, implemented, and validated our controller.
%


%%%%%%%%%%%%%%%%%%%%%%%%%%%%%%%%%%%%%
%%%%%%%%%%%%%%%%%%%%%%%%%%%%%%%%%%%%%
%%%%%%%%%%%%   SECTION   %%%%%%%%%%%%
%%%%%%%%%%%%%%%%%%%%%%%%%%%%%%%%%%%%%
%%%%%%%%%%%%%%%%%%%%%%%%%%%%%%%%%%%%%
\section{Future Directions}


While we have shown significant gains, our experiment setup is limited and does not cover mixed traffic, different packet sizes, or multiple packet flows and distributions.
%
Therefore, additional measurements are necessary to explore the effects of realistic user traffic.
%

%BIBLIOGRAPHY
\addcontentsline{toc}{chapter}{Bibliography}
\bibliographystyle{IEEEtran}
\fancyhead[RE]{\normalfont Bibliography}
\fancyhead[LO]{\normalfont Bibliography}
\bibliography{bibliography/IEEEabrv,bibliography/example}

%ABBILDUNGSVERZEICHNIS
\addcontentsline{toc}{chapter}{List of Figures}
\listoffigures

%TABELLENVERZEICHNIS
\addcontentsline{toc}{chapter}{List of Tables}
\listoftables

%APPENDIX
%\appendix
%\chapter{Appendix}
\label{chap:appendix}


\backmatter
%\include{cv}

\end{document}
